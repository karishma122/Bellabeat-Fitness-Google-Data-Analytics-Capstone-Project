% Options for packages loaded elsewhere
\PassOptionsToPackage{unicode}{hyperref}
\PassOptionsToPackage{hyphens}{url}
%
\documentclass[
]{article}
\usepackage{amsmath,amssymb}
\usepackage{iftex}
\ifPDFTeX
  \usepackage[T1]{fontenc}
  \usepackage[utf8]{inputenc}
  \usepackage{textcomp} % provide euro and other symbols
\else % if luatex or xetex
  \usepackage{unicode-math} % this also loads fontspec
  \defaultfontfeatures{Scale=MatchLowercase}
  \defaultfontfeatures[\rmfamily]{Ligatures=TeX,Scale=1}
\fi
\usepackage{lmodern}
\ifPDFTeX\else
  % xetex/luatex font selection
\fi
% Use upquote if available, for straight quotes in verbatim environments
\IfFileExists{upquote.sty}{\usepackage{upquote}}{}
\IfFileExists{microtype.sty}{% use microtype if available
  \usepackage[]{microtype}
  \UseMicrotypeSet[protrusion]{basicmath} % disable protrusion for tt fonts
}{}
\makeatletter
\@ifundefined{KOMAClassName}{% if non-KOMA class
  \IfFileExists{parskip.sty}{%
    \usepackage{parskip}
  }{% else
    \setlength{\parindent}{0pt}
    \setlength{\parskip}{6pt plus 2pt minus 1pt}}
}{% if KOMA class
  \KOMAoptions{parskip=half}}
\makeatother
\usepackage{xcolor}
\usepackage[margin=1in]{geometry}
\usepackage{color}
\usepackage{fancyvrb}
\newcommand{\VerbBar}{|}
\newcommand{\VERB}{\Verb[commandchars=\\\{\}]}
\DefineVerbatimEnvironment{Highlighting}{Verbatim}{commandchars=\\\{\}}
% Add ',fontsize=\small' for more characters per line
\usepackage{framed}
\definecolor{shadecolor}{RGB}{248,248,248}
\newenvironment{Shaded}{\begin{snugshade}}{\end{snugshade}}
\newcommand{\AlertTok}[1]{\textcolor[rgb]{0.94,0.16,0.16}{#1}}
\newcommand{\AnnotationTok}[1]{\textcolor[rgb]{0.56,0.35,0.01}{\textbf{\textit{#1}}}}
\newcommand{\AttributeTok}[1]{\textcolor[rgb]{0.13,0.29,0.53}{#1}}
\newcommand{\BaseNTok}[1]{\textcolor[rgb]{0.00,0.00,0.81}{#1}}
\newcommand{\BuiltInTok}[1]{#1}
\newcommand{\CharTok}[1]{\textcolor[rgb]{0.31,0.60,0.02}{#1}}
\newcommand{\CommentTok}[1]{\textcolor[rgb]{0.56,0.35,0.01}{\textit{#1}}}
\newcommand{\CommentVarTok}[1]{\textcolor[rgb]{0.56,0.35,0.01}{\textbf{\textit{#1}}}}
\newcommand{\ConstantTok}[1]{\textcolor[rgb]{0.56,0.35,0.01}{#1}}
\newcommand{\ControlFlowTok}[1]{\textcolor[rgb]{0.13,0.29,0.53}{\textbf{#1}}}
\newcommand{\DataTypeTok}[1]{\textcolor[rgb]{0.13,0.29,0.53}{#1}}
\newcommand{\DecValTok}[1]{\textcolor[rgb]{0.00,0.00,0.81}{#1}}
\newcommand{\DocumentationTok}[1]{\textcolor[rgb]{0.56,0.35,0.01}{\textbf{\textit{#1}}}}
\newcommand{\ErrorTok}[1]{\textcolor[rgb]{0.64,0.00,0.00}{\textbf{#1}}}
\newcommand{\ExtensionTok}[1]{#1}
\newcommand{\FloatTok}[1]{\textcolor[rgb]{0.00,0.00,0.81}{#1}}
\newcommand{\FunctionTok}[1]{\textcolor[rgb]{0.13,0.29,0.53}{\textbf{#1}}}
\newcommand{\ImportTok}[1]{#1}
\newcommand{\InformationTok}[1]{\textcolor[rgb]{0.56,0.35,0.01}{\textbf{\textit{#1}}}}
\newcommand{\KeywordTok}[1]{\textcolor[rgb]{0.13,0.29,0.53}{\textbf{#1}}}
\newcommand{\NormalTok}[1]{#1}
\newcommand{\OperatorTok}[1]{\textcolor[rgb]{0.81,0.36,0.00}{\textbf{#1}}}
\newcommand{\OtherTok}[1]{\textcolor[rgb]{0.56,0.35,0.01}{#1}}
\newcommand{\PreprocessorTok}[1]{\textcolor[rgb]{0.56,0.35,0.01}{\textit{#1}}}
\newcommand{\RegionMarkerTok}[1]{#1}
\newcommand{\SpecialCharTok}[1]{\textcolor[rgb]{0.81,0.36,0.00}{\textbf{#1}}}
\newcommand{\SpecialStringTok}[1]{\textcolor[rgb]{0.31,0.60,0.02}{#1}}
\newcommand{\StringTok}[1]{\textcolor[rgb]{0.31,0.60,0.02}{#1}}
\newcommand{\VariableTok}[1]{\textcolor[rgb]{0.00,0.00,0.00}{#1}}
\newcommand{\VerbatimStringTok}[1]{\textcolor[rgb]{0.31,0.60,0.02}{#1}}
\newcommand{\WarningTok}[1]{\textcolor[rgb]{0.56,0.35,0.01}{\textbf{\textit{#1}}}}
\usepackage{graphicx}
\makeatletter
\def\maxwidth{\ifdim\Gin@nat@width>\linewidth\linewidth\else\Gin@nat@width\fi}
\def\maxheight{\ifdim\Gin@nat@height>\textheight\textheight\else\Gin@nat@height\fi}
\makeatother
% Scale images if necessary, so that they will not overflow the page
% margins by default, and it is still possible to overwrite the defaults
% using explicit options in \includegraphics[width, height, ...]{}
\setkeys{Gin}{width=\maxwidth,height=\maxheight,keepaspectratio}
% Set default figure placement to htbp
\makeatletter
\def\fps@figure{htbp}
\makeatother
\setlength{\emergencystretch}{3em} % prevent overfull lines
\providecommand{\tightlist}{%
  \setlength{\itemsep}{0pt}\setlength{\parskip}{0pt}}
\setcounter{secnumdepth}{-\maxdimen} % remove section numbering
\ifLuaTeX
  \usepackage{selnolig}  % disable illegal ligatures
\fi
\IfFileExists{bookmark.sty}{\usepackage{bookmark}}{\usepackage{hyperref}}
\IfFileExists{xurl.sty}{\usepackage{xurl}}{} % add URL line breaks if available
\urlstyle{same}
\hypersetup{
  pdftitle={Bellabeat},
  pdfauthor={Karishma},
  hidelinks,
  pdfcreator={LaTeX via pandoc}}

\title{Bellabeat}
\author{Karishma}
\date{2023-12-20}

\begin{document}
\maketitle

\hypertarget{introduction}{%
\subsection{Introduction}\label{introduction}}

Bellabeat is a high-tech manufacturer of health-focused products for
women. Collecting data on activity, sleep, stress, and reproductive
health has allowed Bellabeat to empower women with knowledge about their
own health and habits. Although Bellabeat is a successful small company,
they have the potential to become a larger player in the global smart
device market. Urška Sršen, cofounder and Chief Creative Officer of
Bellabeat, believes that analyzing smart device fitness data could help
unlock new growth opportunities for the company.

\hypertarget{ask}{%
\subsection{1. Ask}\label{ask}}

\hypertarget{business-task}{%
\subparagraph{Business Task:}\label{business-task}}

To identify potential opportunities for growth and provide
recommendations for the Bellabeat marketing strategy improvement based
on trends in smart device usage.

\hypertarget{key-stakeholders}{%
\subparagraph{Key Stakeholders:}\label{key-stakeholders}}

\begin{itemize}
\tightlist
\item
  Urška Sršen: Bellabeat's cofounder and Chief Creative Officer
\item
  Sando Mur: Mathematician and Bellabeat's co-founder
\end{itemize}

\hypertarget{questions-to-explore-for-the-analysis}{%
\subparagraph{Questions to explore for the
analysis:}\label{questions-to-explore-for-the-analysis}}

\begin{itemize}
\tightlist
\item
  What are some trends in smart device usage?
\item
  How could these trends apply to Bellabeat customers?
\item
  How could these trends help influence Bellabeat marketing strategy?
\end{itemize}

\hypertarget{prepare}{%
\subsection{2. Prepare}\label{prepare}}

This Kaggle data set contains personal fitness tracker from thirty
fitbit users. Thirty eligible Fitbit users consented to the submission
of personal tracker data, including minute-level output for physical
activity, heart rate, and sleep monitoring. It includes information
about daily activity, steps, and heart rate that can be used to explore
users' habits.

\hypertarget{loading-packages}{%
\paragraph{Loading Packages}\label{loading-packages}}

\begin{Shaded}
\begin{Highlighting}[]
\FunctionTok{library}\NormalTok{(tidyverse)}
\end{Highlighting}
\end{Shaded}

\begin{verbatim}
## -- Attaching core tidyverse packages ------------------------ tidyverse 2.0.0 --
## v dplyr     1.1.4     v readr     2.1.4
## v forcats   1.0.0     v stringr   1.5.1
## v ggplot2   3.4.4     v tibble    3.2.1
## v lubridate 1.9.3     v tidyr     1.3.0
## v purrr     1.0.2     
## -- Conflicts ------------------------------------------ tidyverse_conflicts() --
## x dplyr::filter() masks stats::filter()
## x dplyr::lag()    masks stats::lag()
## i Use the conflicted package (<http://conflicted.r-lib.org/>) to force all conflicts to become errors
\end{verbatim}

\begin{Shaded}
\begin{Highlighting}[]
\FunctionTok{library}\NormalTok{(lubridate) }
\FunctionTok{library}\NormalTok{(dplyr)}
\FunctionTok{library}\NormalTok{(ggplot2)}
\FunctionTok{library}\NormalTok{(tidyr)}
\FunctionTok{library}\NormalTok{(janitor)}
\end{Highlighting}
\end{Shaded}

\begin{verbatim}
## 
## Attaching package: 'janitor'
## 
## The following objects are masked from 'package:stats':
## 
##     chisq.test, fisher.test
\end{verbatim}

\hypertarget{process}{%
\subsection{3. Process}\label{process}}

\hypertarget{importing-the-datasets}{%
\paragraph{Importing the Datasets}\label{importing-the-datasets}}

\begin{Shaded}
\begin{Highlighting}[]
\CommentTok{\# Read the dataframes}
\NormalTok{activity }\OtherTok{\textless{}{-}} \FunctionTok{read\_csv}\NormalTok{(}\StringTok{"C:/Users/karis/Downloads/input/Fitabase Data 4.12.16{-}5.12.16/dailyActivity\_merged.csv"}\NormalTok{)}
\end{Highlighting}
\end{Shaded}

\begin{verbatim}
## Rows: 940 Columns: 15
## -- Column specification --------------------------------------------------------
## Delimiter: ","
## chr  (1): ActivityDate
## dbl (14): Id, TotalSteps, TotalDistance, TrackerDistance, LoggedActivitiesDi...
## 
## i Use `spec()` to retrieve the full column specification for this data.
## i Specify the column types or set `show_col_types = FALSE` to quiet this message.
\end{verbatim}

\begin{Shaded}
\begin{Highlighting}[]
\NormalTok{calories }\OtherTok{\textless{}{-}} \FunctionTok{read\_csv}\NormalTok{(}\StringTok{"C:/Users/karis/Downloads/input/Fitabase Data 4.12.16{-}5.12.16/dailyCalories\_merged.csv"}\NormalTok{)}
\end{Highlighting}
\end{Shaded}

\begin{verbatim}
## Rows: 940 Columns: 3
## -- Column specification --------------------------------------------------------
## Delimiter: ","
## chr (1): ActivityDay
## dbl (2): Id, Calories
## 
## i Use `spec()` to retrieve the full column specification for this data.
## i Specify the column types or set `show_col_types = FALSE` to quiet this message.
\end{verbatim}

\begin{Shaded}
\begin{Highlighting}[]
\NormalTok{intensities }\OtherTok{\textless{}{-}} \FunctionTok{read\_csv}\NormalTok{(}\StringTok{"C:/Users/karis/Downloads/input/Fitabase Data 4.12.16{-}5.12.16/hourlyIntensities\_merged.csv"}\NormalTok{)}
\end{Highlighting}
\end{Shaded}

\begin{verbatim}
## Rows: 22099 Columns: 4
## -- Column specification --------------------------------------------------------
## Delimiter: ","
## chr (1): ActivityHour
## dbl (3): Id, TotalIntensity, AverageIntensity
## 
## i Use `spec()` to retrieve the full column specification for this data.
## i Specify the column types or set `show_col_types = FALSE` to quiet this message.
\end{verbatim}

\begin{Shaded}
\begin{Highlighting}[]
\NormalTok{sleep }\OtherTok{\textless{}{-}} \FunctionTok{read\_csv}\NormalTok{(}\StringTok{"C:/Users/karis/Downloads/input/Fitabase Data 4.12.16{-}5.12.16/sleepDay\_merged.csv"}\NormalTok{)}
\end{Highlighting}
\end{Shaded}

\begin{verbatim}
## Rows: 413 Columns: 5
## -- Column specification --------------------------------------------------------
## Delimiter: ","
## chr (1): SleepDay
## dbl (4): Id, TotalSleepRecords, TotalMinutesAsleep, TotalTimeInBed
## 
## i Use `spec()` to retrieve the full column specification for this data.
## i Specify the column types or set `show_col_types = FALSE` to quiet this message.
\end{verbatim}

\begin{Shaded}
\begin{Highlighting}[]
\NormalTok{weight }\OtherTok{\textless{}{-}} \FunctionTok{read\_csv}\NormalTok{(}\StringTok{"C:/Users/karis/Downloads/input/Fitabase Data 4.12.16{-}5.12.16/weightLogInfo\_merged.csv"}\NormalTok{)}
\end{Highlighting}
\end{Shaded}

\begin{verbatim}
## Rows: 67 Columns: 8
## -- Column specification --------------------------------------------------------
## Delimiter: ","
## chr (1): Date
## dbl (6): Id, WeightKg, WeightPounds, Fat, BMI, LogId
## lgl (1): IsManualReport
## 
## i Use `spec()` to retrieve the full column specification for this data.
## i Specify the column types or set `show_col_types = FALSE` to quiet this message.
\end{verbatim}

\hypertarget{data}{%
\paragraph{data}\label{data}}

\begin{Shaded}
\begin{Highlighting}[]
\FunctionTok{head}\NormalTok{(activity)}
\end{Highlighting}
\end{Shaded}

\begin{verbatim}
## # A tibble: 6 x 15
##           Id ActivityDate TotalSteps TotalDistance TrackerDistance
##        <dbl> <chr>             <dbl>         <dbl>           <dbl>
## 1 1503960366 4/12/2016         13162          8.5             8.5 
## 2 1503960366 4/13/2016         10735          6.97            6.97
## 3 1503960366 4/14/2016         10460          6.74            6.74
## 4 1503960366 4/15/2016          9762          6.28            6.28
## 5 1503960366 4/16/2016         12669          8.16            8.16
## 6 1503960366 4/17/2016          9705          6.48            6.48
## # i 10 more variables: LoggedActivitiesDistance <dbl>,
## #   VeryActiveDistance <dbl>, ModeratelyActiveDistance <dbl>,
## #   LightActiveDistance <dbl>, SedentaryActiveDistance <dbl>,
## #   VeryActiveMinutes <dbl>, FairlyActiveMinutes <dbl>,
## #   LightlyActiveMinutes <dbl>, SedentaryMinutes <dbl>, Calories <dbl>
\end{verbatim}

\begin{Shaded}
\begin{Highlighting}[]
\FunctionTok{colnames}\NormalTok{(activity)}
\end{Highlighting}
\end{Shaded}

\begin{verbatim}
##  [1] "Id"                       "ActivityDate"            
##  [3] "TotalSteps"               "TotalDistance"           
##  [5] "TrackerDistance"          "LoggedActivitiesDistance"
##  [7] "VeryActiveDistance"       "ModeratelyActiveDistance"
##  [9] "LightActiveDistance"      "SedentaryActiveDistance" 
## [11] "VeryActiveMinutes"        "FairlyActiveMinutes"     
## [13] "LightlyActiveMinutes"     "SedentaryMinutes"        
## [15] "Calories"
\end{verbatim}

\begin{Shaded}
\begin{Highlighting}[]
\FunctionTok{head}\NormalTok{(weight)}
\end{Highlighting}
\end{Shaded}

\begin{verbatim}
## # A tibble: 6 x 8
##           Id Date       WeightKg WeightPounds   Fat   BMI IsManualReport   LogId
##        <dbl> <chr>         <dbl>        <dbl> <dbl> <dbl> <lgl>            <dbl>
## 1 1503960366 5/2/2016 ~     52.6         116.    22  22.6 TRUE           1.46e12
## 2 1503960366 5/3/2016 ~     52.6         116.    NA  22.6 TRUE           1.46e12
## 3 1927972279 4/13/2016~    134.          294.    NA  47.5 FALSE          1.46e12
## 4 2873212765 4/21/2016~     56.7         125.    NA  21.5 TRUE           1.46e12
## 5 2873212765 5/12/2016~     57.3         126.    NA  21.7 TRUE           1.46e12
## 6 4319703577 4/17/2016~     72.4         160.    25  27.5 TRUE           1.46e12
\end{verbatim}

\begin{Shaded}
\begin{Highlighting}[]
\FunctionTok{colnames}\NormalTok{(weight)}
\end{Highlighting}
\end{Shaded}

\begin{verbatim}
## [1] "Id"             "Date"           "WeightKg"       "WeightPounds"  
## [5] "Fat"            "BMI"            "IsManualReport" "LogId"
\end{verbatim}

\hypertarget{converting-date-time-format}{%
\paragraph{Converting date time
format}\label{converting-date-time-format}}

\begin{Shaded}
\begin{Highlighting}[]
\CommentTok{\# intensities}
\NormalTok{intensities}\SpecialCharTok{$}\NormalTok{ActivityHour}\OtherTok{=}\FunctionTok{as.POSIXct}\NormalTok{(intensities}\SpecialCharTok{$}\NormalTok{ActivityHour, }\AttributeTok{format=}\StringTok{"\%m/\%d/\%Y \%I:\%M:\%S \%p"}\NormalTok{, }\AttributeTok{tz=}\FunctionTok{Sys.timezone}\NormalTok{())}
\NormalTok{intensities}\SpecialCharTok{$}\NormalTok{time }\OtherTok{\textless{}{-}} \FunctionTok{format}\NormalTok{(intensities}\SpecialCharTok{$}\NormalTok{ActivityHour, }\AttributeTok{format =} \StringTok{"\%H:\%M:\%S"}\NormalTok{)}
\NormalTok{intensities}\SpecialCharTok{$}\NormalTok{date }\OtherTok{\textless{}{-}} \FunctionTok{format}\NormalTok{(intensities}\SpecialCharTok{$}\NormalTok{ActivityHour, }\AttributeTok{format =} \StringTok{"\%m/\%d/\%y"}\NormalTok{)}
\CommentTok{\# activity}
\NormalTok{activity}\SpecialCharTok{$}\NormalTok{ActivityDate}\OtherTok{=}\FunctionTok{as.POSIXct}\NormalTok{(activity}\SpecialCharTok{$}\NormalTok{ActivityDate, }\AttributeTok{format=}\StringTok{"\%m/\%d/\%Y"}\NormalTok{, }\AttributeTok{tz=}\FunctionTok{Sys.timezone}\NormalTok{())}
\NormalTok{activity}\SpecialCharTok{$}\NormalTok{date }\OtherTok{\textless{}{-}} \FunctionTok{format}\NormalTok{(activity}\SpecialCharTok{$}\NormalTok{ActivityDate, }\AttributeTok{format =} \StringTok{"\%m/\%d/\%y"}\NormalTok{)}
\CommentTok{\# sleep}
\NormalTok{sleep}\SpecialCharTok{$}\NormalTok{SleepDay}\OtherTok{=}\FunctionTok{as.POSIXct}\NormalTok{(sleep}\SpecialCharTok{$}\NormalTok{SleepDay, }\AttributeTok{format=}\StringTok{"\%m/\%d/\%Y \%I:\%M:\%S \%p"}\NormalTok{, }\AttributeTok{tz=}\FunctionTok{Sys.timezone}\NormalTok{())}
\NormalTok{sleep}\SpecialCharTok{$}\NormalTok{date }\OtherTok{\textless{}{-}} \FunctionTok{format}\NormalTok{(sleep}\SpecialCharTok{$}\NormalTok{SleepDay, }\AttributeTok{format =} \StringTok{"\%m/\%d/\%y"}\NormalTok{)}
\end{Highlighting}
\end{Shaded}

\hypertarget{analyze}{%
\subsection{4. Analyze}\label{analyze}}

\hypertarget{number-of-participants-in-each-category}{%
\paragraph{Number of Participants in each
category}\label{number-of-participants-in-each-category}}

\begin{Shaded}
\begin{Highlighting}[]
\FunctionTok{n\_distinct}\NormalTok{(activity}\SpecialCharTok{$}\NormalTok{Id)  }
\end{Highlighting}
\end{Shaded}

\begin{verbatim}
## [1] 33
\end{verbatim}

\begin{Shaded}
\begin{Highlighting}[]
\FunctionTok{n\_distinct}\NormalTok{(calories}\SpecialCharTok{$}\NormalTok{Id)   }
\end{Highlighting}
\end{Shaded}

\begin{verbatim}
## [1] 33
\end{verbatim}

\begin{Shaded}
\begin{Highlighting}[]
\FunctionTok{n\_distinct}\NormalTok{(intensities}\SpecialCharTok{$}\NormalTok{Id)}
\end{Highlighting}
\end{Shaded}

\begin{verbatim}
## [1] 33
\end{verbatim}

\begin{Shaded}
\begin{Highlighting}[]
\FunctionTok{n\_distinct}\NormalTok{(sleep}\SpecialCharTok{$}\NormalTok{Id)}
\end{Highlighting}
\end{Shaded}

\begin{verbatim}
## [1] 24
\end{verbatim}

\begin{Shaded}
\begin{Highlighting}[]
\FunctionTok{n\_distinct}\NormalTok{(weight}\SpecialCharTok{$}\NormalTok{Id)}
\end{Highlighting}
\end{Shaded}

\begin{verbatim}
## [1] 8
\end{verbatim}

To summarize the above data, there are 33 participants in the activity,
calories, and intensities datasets, 24 in the sleep dataset, and only 8
in the weight dataset. The fact that there are only 8 participants in
the weight dataset means that more data would be needed to make a strong
reccomendation or conclusion.

\hypertarget{checking-for-significant-change-in-weight}{%
\paragraph{checking for significant change in
weight}\label{checking-for-significant-change-in-weight}}

\begin{Shaded}
\begin{Highlighting}[]
\NormalTok{weight}\SpecialCharTok{\%\textgreater{}\%}
 \FunctionTok{group\_by}\NormalTok{(Id)}\SpecialCharTok{\%\textgreater{}\%}
  \FunctionTok{summarise}\NormalTok{(}\FunctionTok{min}\NormalTok{(WeightKg),}\FunctionTok{max}\NormalTok{(WeightKg))}
\end{Highlighting}
\end{Shaded}

\begin{verbatim}
## # A tibble: 8 x 3
##           Id `min(WeightKg)` `max(WeightKg)`
##        <dbl>           <dbl>           <dbl>
## 1 1503960366            52.6            52.6
## 2 1927972279           134.            134. 
## 3 2873212765            56.7            57.3
## 4 4319703577            72.3            72.4
## 5 4558609924            69.1            70.3
## 6 5577150313            90.7            90.7
## 7 6962181067            61              62.5
## 8 8877689391            84              85.8
\end{verbatim}

There is no significant changes in weight of 8 participants.

\hypertarget{the-summaries-for-the-rest-of-the-datasets}{%
\paragraph{The summaries for the rest of the
datasets:}\label{the-summaries-for-the-rest-of-the-datasets}}

\begin{Shaded}
\begin{Highlighting}[]
\CommentTok{\# activity}
\NormalTok{activity }\SpecialCharTok{\%\textgreater{}\%}  
  \FunctionTok{select}\NormalTok{(TotalSteps,}
\NormalTok{         TotalDistance,}
\NormalTok{         SedentaryMinutes, Calories) }\SpecialCharTok{\%\textgreater{}\%}
  \FunctionTok{summary}\NormalTok{()}
\end{Highlighting}
\end{Shaded}

\begin{verbatim}
##    TotalSteps    TotalDistance    SedentaryMinutes    Calories   
##  Min.   :    0   Min.   : 0.000   Min.   :   0.0   Min.   :   0  
##  1st Qu.: 3790   1st Qu.: 2.620   1st Qu.: 729.8   1st Qu.:1828  
##  Median : 7406   Median : 5.245   Median :1057.5   Median :2134  
##  Mean   : 7638   Mean   : 5.490   Mean   : 991.2   Mean   :2304  
##  3rd Qu.:10727   3rd Qu.: 7.713   3rd Qu.:1229.5   3rd Qu.:2793  
##  Max.   :36019   Max.   :28.030   Max.   :1440.0   Max.   :4900
\end{verbatim}

\begin{Shaded}
\begin{Highlighting}[]
\CommentTok{\# active minutes per category}
\NormalTok{activity }\SpecialCharTok{\%\textgreater{}\%}
  \FunctionTok{select}\NormalTok{(VeryActiveMinutes, FairlyActiveMinutes, LightlyActiveMinutes) }\SpecialCharTok{\%\textgreater{}\%}
  \FunctionTok{summary}\NormalTok{()}
\end{Highlighting}
\end{Shaded}

\begin{verbatim}
##  VeryActiveMinutes FairlyActiveMinutes LightlyActiveMinutes
##  Min.   :  0.00    Min.   :  0.00      Min.   :  0.0       
##  1st Qu.:  0.00    1st Qu.:  0.00      1st Qu.:127.0       
##  Median :  4.00    Median :  6.00      Median :199.0       
##  Mean   : 21.16    Mean   : 13.56      Mean   :192.8       
##  3rd Qu.: 32.00    3rd Qu.: 19.00      3rd Qu.:264.0       
##  Max.   :210.00    Max.   :143.00      Max.   :518.0
\end{verbatim}

\begin{Shaded}
\begin{Highlighting}[]
\CommentTok{\# calories}
\NormalTok{calories }\SpecialCharTok{\%\textgreater{}\%}
  \FunctionTok{select}\NormalTok{(Calories) }\SpecialCharTok{\%\textgreater{}\%}
  \FunctionTok{summary}\NormalTok{()}
\end{Highlighting}
\end{Shaded}

\begin{verbatim}
##     Calories   
##  Min.   :   0  
##  1st Qu.:1828  
##  Median :2134  
##  Mean   :2304  
##  3rd Qu.:2793  
##  Max.   :4900
\end{verbatim}

\begin{Shaded}
\begin{Highlighting}[]
\CommentTok{\# sleep}
\NormalTok{sleep }\SpecialCharTok{\%\textgreater{}\%}
  \FunctionTok{select}\NormalTok{(TotalSleepRecords, TotalMinutesAsleep, TotalTimeInBed) }\SpecialCharTok{\%\textgreater{}\%}
  \FunctionTok{summary}\NormalTok{()}
\end{Highlighting}
\end{Shaded}

\begin{verbatim}
##  TotalSleepRecords TotalMinutesAsleep TotalTimeInBed 
##  Min.   :1.000     Min.   : 58.0      Min.   : 61.0  
##  1st Qu.:1.000     1st Qu.:361.0      1st Qu.:403.0  
##  Median :1.000     Median :433.0      Median :463.0  
##  Mean   :1.119     Mean   :419.5      Mean   :458.6  
##  3rd Qu.:1.000     3rd Qu.:490.0      3rd Qu.:526.0  
##  Max.   :3.000     Max.   :796.0      Max.   :961.0
\end{verbatim}

\begin{Shaded}
\begin{Highlighting}[]
\CommentTok{\# weight}
\NormalTok{weight }\SpecialCharTok{\%\textgreater{}\%}
  \FunctionTok{select}\NormalTok{(WeightKg, BMI) }\SpecialCharTok{\%\textgreater{}\%}
  \FunctionTok{summary}\NormalTok{()}
\end{Highlighting}
\end{Shaded}

\begin{verbatim}
##     WeightKg           BMI       
##  Min.   : 52.60   Min.   :21.45  
##  1st Qu.: 61.40   1st Qu.:23.96  
##  Median : 62.50   Median :24.39  
##  Mean   : 72.04   Mean   :25.19  
##  3rd Qu.: 85.05   3rd Qu.:25.56  
##  Max.   :133.50   Max.   :47.54
\end{verbatim}

\hypertarget{observations-made-from-the-above-summaries}{%
\paragraph{Observations made from the above
summaries:}\label{observations-made-from-the-above-summaries}}

\begin{itemize}
\tightlist
\item
  Sedetary minutes on average is 16.5 hours.
\item
  The average number of steps per day is 7638. The CDC recommends people
  take 10,000 steps daily.
\item
  The majority of the participants are lightly active.
\item
  The average participant burns 97 calories per hour.
\item
  On an average, participants sleep for 7 hours.
\end{itemize}

\hypertarget{merging-data}{%
\paragraph{Merging Data}\label{merging-data}}

Merging two datasets Activity and Sleep on Columns Id and date.

\begin{Shaded}
\begin{Highlighting}[]
\NormalTok{merged\_data }\OtherTok{\textless{}{-}} \FunctionTok{merge}\NormalTok{(sleep, activity, }\AttributeTok{by =} \FunctionTok{c}\NormalTok{(}\StringTok{\textquotesingle{}Id\textquotesingle{}}\NormalTok{, }\StringTok{\textquotesingle{}date\textquotesingle{}}\NormalTok{))}
\FunctionTok{head}\NormalTok{(merged\_data) }
\end{Highlighting}
\end{Shaded}

\begin{verbatim}
##           Id     date   SleepDay TotalSleepRecords TotalMinutesAsleep
## 1 1503960366 04/12/16 2016-04-12                 1                327
## 2 1503960366 04/13/16 2016-04-13                 2                384
## 3 1503960366 04/15/16 2016-04-15                 1                412
## 4 1503960366 04/16/16 2016-04-16                 2                340
## 5 1503960366 04/17/16 2016-04-17                 1                700
## 6 1503960366 04/19/16 2016-04-19                 1                304
##   TotalTimeInBed ActivityDate TotalSteps TotalDistance TrackerDistance
## 1            346   2016-04-12      13162          8.50            8.50
## 2            407   2016-04-13      10735          6.97            6.97
## 3            442   2016-04-15       9762          6.28            6.28
## 4            367   2016-04-16      12669          8.16            8.16
## 5            712   2016-04-17       9705          6.48            6.48
## 6            320   2016-04-19      15506          9.88            9.88
##   LoggedActivitiesDistance VeryActiveDistance ModeratelyActiveDistance
## 1                        0               1.88                     0.55
## 2                        0               1.57                     0.69
## 3                        0               2.14                     1.26
## 4                        0               2.71                     0.41
## 5                        0               3.19                     0.78
## 6                        0               3.53                     1.32
##   LightActiveDistance SedentaryActiveDistance VeryActiveMinutes
## 1                6.06                       0                25
## 2                4.71                       0                21
## 3                2.83                       0                29
## 4                5.04                       0                36
## 5                2.51                       0                38
## 6                5.03                       0                50
##   FairlyActiveMinutes LightlyActiveMinutes SedentaryMinutes Calories
## 1                  13                  328              728     1985
## 2                  19                  217              776     1797
## 3                  34                  209              726     1745
## 4                  10                  221              773     1863
## 5                  20                  164              539     1728
## 6                  31                  264              775     2035
\end{verbatim}

\hypertarget{share}{%
\subsection{5. Share}\label{share}}

\begin{Shaded}
\begin{Highlighting}[]
\FunctionTok{ggplot}\NormalTok{(}\AttributeTok{data =}\NormalTok{ activity, }\FunctionTok{aes}\NormalTok{(}\AttributeTok{x =}\NormalTok{ TotalSteps, }\AttributeTok{y =}\NormalTok{ Calories)) }\SpecialCharTok{+} \FunctionTok{geom\_point}\NormalTok{() }\SpecialCharTok{+} \FunctionTok{geom\_smooth}\NormalTok{() }\SpecialCharTok{+} \FunctionTok{labs}\NormalTok{(}\AttributeTok{title =} \StringTok{"Total Steps vs. Calories"}\NormalTok{)}
\end{Highlighting}
\end{Shaded}

\begin{verbatim}
## `geom_smooth()` using method = 'loess' and formula = 'y ~ x'
\end{verbatim}

\includegraphics{Bellabeat_files/figure-latex/unnamed-chunk-12-1.pdf}
There is a correlation between total number of steps taken and calories
burned. The more steps each participant takes, the more calories they
burn.

\begin{Shaded}
\begin{Highlighting}[]
\FunctionTok{ggplot}\NormalTok{(}\AttributeTok{data =}\NormalTok{ sleep, }\FunctionTok{aes}\NormalTok{(}\AttributeTok{x =}\NormalTok{ TotalMinutesAsleep, }\AttributeTok{y =}\NormalTok{ TotalTimeInBed)) }\SpecialCharTok{+} \FunctionTok{geom\_point}\NormalTok{() }\SpecialCharTok{+} \FunctionTok{labs}\NormalTok{(}\AttributeTok{title =} \StringTok{"Total time asleep vs Total time in bed"}\NormalTok{)}
\end{Highlighting}
\end{Shaded}

\includegraphics{Bellabeat_files/figure-latex/unnamed-chunk-13-1.pdf}
There is a positive correlation between total time asleep vs total time
in bed. To improve sleep quality for its users, bellabeat should
consider having a section where users can customize their sleep schedule
to ensure consistency.

\begin{Shaded}
\begin{Highlighting}[]
\FunctionTok{ggplot}\NormalTok{(}\AttributeTok{data =}\NormalTok{ merged\_data, }\AttributeTok{mapping =} \FunctionTok{aes}\NormalTok{(}\AttributeTok{x =}\NormalTok{ SedentaryMinutes, }\AttributeTok{y =}\NormalTok{ TotalMinutesAsleep)) }\SpecialCharTok{+} 
  \FunctionTok{geom\_point}\NormalTok{() }\SpecialCharTok{+} \FunctionTok{labs}\NormalTok{(}\AttributeTok{title=} \StringTok{"Sleep Duration and Sedentary Time"}\NormalTok{)}
\end{Highlighting}
\end{Shaded}

\includegraphics{Bellabeat_files/figure-latex/unnamed-chunk-14-1.pdf}

\begin{Shaded}
\begin{Highlighting}[]
\FunctionTok{cor}\NormalTok{(merged\_data}\SpecialCharTok{$}\NormalTok{TotalMinutesAsleep,merged\_data}\SpecialCharTok{$}\NormalTok{SedentaryMinutes)}
\end{Highlighting}
\end{Shaded}

\begin{verbatim}
## [1] -0.599394
\end{verbatim}

There is a negative correlation between SedentaryMinutes and
TotalMinutesAsleep. This means that the less active a participant is,
the less sleep they tend to get.

\hypertarget{whether-the-day-of-the-week-affects-our-activity-levels-and-sleep.}{%
\paragraph{Whether the day of the week affects our activity levels and
sleep.}\label{whether-the-day-of-the-week-affects-our-activity-levels-and-sleep.}}

\begin{Shaded}
\begin{Highlighting}[]
\CommentTok{\# aggregate data by day of week to summarize averages }
\NormalTok{merged\_data }\OtherTok{\textless{}{-}} \FunctionTok{mutate}\NormalTok{(merged\_data,}\AttributeTok{day =} \FunctionTok{wday}\NormalTok{(SleepDay, }\AttributeTok{label =} \ConstantTok{TRUE}\NormalTok{))}
\NormalTok{summarized\_activity\_sleep }\OtherTok{\textless{}{-}}\NormalTok{ merged\_data }\SpecialCharTok{\%\textgreater{}\%} 
  \FunctionTok{group\_by}\NormalTok{(day) }\SpecialCharTok{\%\textgreater{}\%} 
  \FunctionTok{summarise}\NormalTok{(}\AttributeTok{AvgDailySteps =} \FunctionTok{mean}\NormalTok{(TotalSteps),}
            \AttributeTok{AvgAsleepMinutes =} \FunctionTok{mean}\NormalTok{(TotalMinutesAsleep),}
            \AttributeTok{AvgAwakeTimeInBed =} \FunctionTok{mean}\NormalTok{(TotalTimeInBed), }
            \AttributeTok{AvgSedentaryMinutes =} \FunctionTok{mean}\NormalTok{(SedentaryMinutes),}
            \AttributeTok{AvgLightlyActiveMinutes =} \FunctionTok{mean}\NormalTok{(LightlyActiveMinutes),}
            \AttributeTok{AvgFairlyActiveMinutes =} \FunctionTok{mean}\NormalTok{(FairlyActiveMinutes),}
            \AttributeTok{AvgVeryActiveMinutes =} \FunctionTok{mean}\NormalTok{(VeryActiveMinutes), }
            \AttributeTok{AvgCalories =} \FunctionTok{mean}\NormalTok{(Calories))}
\FunctionTok{head}\NormalTok{(summarized\_activity\_sleep)}
\end{Highlighting}
\end{Shaded}

\begin{verbatim}
## # A tibble: 6 x 9
##   day   AvgDailySteps AvgAsleepMinutes AvgAwakeTimeInBed AvgSedentaryMinutes
##   <ord>         <dbl>            <dbl>             <dbl>               <dbl>
## 1 Sun           7298.             453.              504.                688.
## 2 Mon           9340.             419.              456.                718.
## 3 Tue           9183.             405.              443.                740.
## 4 Wed           8023.             435.              470.                714.
## 5 Thu           8205.             402.              436.                701.
## 6 Fri           7901.             405.              445.                743.
## # i 4 more variables: AvgLightlyActiveMinutes <dbl>,
## #   AvgFairlyActiveMinutes <dbl>, AvgVeryActiveMinutes <dbl>, AvgCalories <dbl>
\end{verbatim}

\begin{Shaded}
\begin{Highlighting}[]
\FunctionTok{ggplot}\NormalTok{(}\AttributeTok{data =}\NormalTok{ summarized\_activity\_sleep, }\AttributeTok{mapping =} \FunctionTok{aes}\NormalTok{(}\AttributeTok{x =}\NormalTok{ day, }\AttributeTok{y =}\NormalTok{ AvgDailySteps)) }\SpecialCharTok{+}
\FunctionTok{geom\_col}\NormalTok{(}\AttributeTok{fill =} \StringTok{"green"}\NormalTok{) }\SpecialCharTok{+} \FunctionTok{labs}\NormalTok{(}\AttributeTok{title =} \StringTok{"Daily Step Count"}\NormalTok{)}
\end{Highlighting}
\end{Shaded}

\includegraphics{Bellabeat_files/figure-latex/unnamed-chunk-17-1.pdf}
The bar graph above shows us that participants are most active on
saturdays and least active on sundays.

\hypertarget{act}{%
\subsection{6. Act}\label{act}}

After analyzing the FitBit Fitness Tracker data, I came up with some
recommendations for Bellabeat marketing strategy based on trends in
smart device usage.

\begin{itemize}
\tightlist
\item
  The majority of participants are lightly active. Bellabeat should
  offer a progression system in the app to encourage participants to
  become at least fairly active.
\item
  If users want to improve the quality of their sleep, Bellabeat should
  consider using app notifications reminding users to get enough rest,
  as well as recommending reducing sedentary time.
\item
  Participants are most active on Saturdays. Bellabeat can use this
  knowledge to remind users to go for a walk or a jog on that day.
  Participants seem to be the least active on Sundays. Bellabeat can use
  this to motivate users to go out and continue exercising on Sundays.
\end{itemize}

\end{document}
